\ifthesis@glossary\else
   \thesis@info{Disable automatic glossary configuration (disabled)}
   \endinput
\fi

\thesis@info{Apply glossary configuration}

% TODO: clean up, heavily based on my bachelor thesis configuration

\RequirePackage[
   acronym,
   nopostdot,
   numberedsection=nameref,
   section,
   noglossaryindex,
   translate=babel
]{glossaries-accsupp}

% \noist % disable index style file
\RequirePackage{glossary-longragged}
% TODO, ensure? \RequirePackage{enumitem}

\newif\ifdisableside

\newif\if@lglos@atinit@glossary@
% single- and plural-page
% this is a hack. i check if 'delimN' or 'delimR' are in the page def
% therefore i execute the page-commands but gobble all of their outputs
\newsavebox\@lglos@la@void
\def\@lglos@hide@in@da@void#1{\sbox\@lglos@la@void{#1}}

% TODO: translate
\newif\iflglos@thispage@issingle@
\let\lglos@legacy@delimN\delimN
\let\lglos@legacy@delimR\delimR

\def\lglos@delimN{%
   \global\lglos@thispage@issingle@false\lglos@legacy@delimN
}
\def\lglos@delimR{%
   \global\lglos@thispage@issingle@false\lglos@legacy@delimN
}
% #1 if lglos@spage or @ppage, #2 content
\def\@glos@strutinit{\vspace{\dimexpr-\ht\strutbox+\dp\strutbox}\thesis@marginpar@font\scriptsize}
\def\@@lglos@printpages#1#2{%
\sidenote*{\@glos@strutinit#1:~\nohyper{#2.}%
}%
}
\def\@lglos@printpages#1{%
   \protected@edef\@tmp@curpage@pagename{\iflglos@thispage@issingle@\lglos@spage\else\lglos@ppage\fi}%
   \expandafter\@@lglos@printpages\expandafter{\@tmp@curpage@pagename}{#1}%
}
\def\lglos@printpages#1{%
   \let\delimN\lglos@delimN\let\delimR\lglos@delimR
   \lglos@thispage@issingle@true\@lglos@hide@in@da@void{#1}%
   \@lglos@printpages{#1}%
}

\long\def\SetTheGlossaryStyle#1{%
   \long\gdef\lglos@headingstyle##1{#1{##1}}%
}

\def\lglos@default@headingstyle#1{%
   \begin{center}%
      \strut\emph{#1}%
   \end{center}%
}
\def\lglos@default@headingstyle@short#1{\medskip}

\def\lglos@desc@options{}

\newglossarystyle{lecture@glossary}{%
   \setglossarystyle{tree}%
   \renewcommand\glsgroupheading[1]{%
      \if@lglos@atinit@glossary@\else
         \def\lglos@enddesc{\end{description}}%
         \expandafter\lglos@enddesc
      \fi
      \global\@lglos@atinit@glossary@false
      \lglos@headingstyle{\glsgetgrouptitle{##1}}%
      \protected@edef\lglos@curdesc{\noexpand\begin{description}[\lglos@desc@options]}\expandafter\lglos@curdesc
   }%
   \renewenvironment{theglossary}{%
      \parindent\z@ \parskip\z@
      \global\@lglos@atinit@glossary@true
   }{\end{description}}%#
   \renewcommand\glossentry[2]{%reset of tree to pad pagenumbers
      \hangindent\z@ \parindent\z@
      \item[\glsentryitem{##1}\glstreenamefmt{\glstarget{##1}{\glossentryname{##1}}}]%
      \strut\lglos@printpages{##2}\ifglshassymbol{##1}{\space(\glossentrysymbol{##1})}{}%
      \glstreepredesc\glossentrydesc{##1}\glspostdescription{}\par
   }%
}

\setglossarystyle{lecture@glossary}

\AtEndPreamble{\makeglossaries}% so we can have new types etc.
\preto\printglossaries{\begingroup\thesisglsinsidefalse\interlinepenalty\@M}
\appto\printglossaries{\endgroup}

\newrobustcmd*\glslong[1]{{\glsreset{#1}\gls{#1}}}\newrobustcmd*\glspllong[1]{{\glsreset{#1}\glspl{#1}}}
\newrobustcmd*\glsshort[1]{{\glsunset{#1}\gls{#1}}}\newrobustcmd*\glsplshort[1]{{\glsunset{#1}\glspl{#1}}}
\newrobustcmd*\Glslong[1]{{\glsreset{#1}\Gls{#1}}}\newrobustcmd*\Glspllong[1]{{\glsreset{#1}\Glspl{#1}}}
\newrobustcmd*\Glsshort[1]{{\glsunset{#1}\Gls{#1}}}\newrobustcmd*\Glsplshort[1]{{\glsunset{#1}\Glspl{#1}}}

% will be used in sidebar notes
\glsaddstoragekey{shortdesc}{\glsentrydesc{\glslabel}}{\glsshortentrydesc}
\def\thesis@shortgls{\protect\glsshortentrydesc{\glslabel}}

% change gls base; does never print long for ac;
\newif\ifglsshouldbeit \glsshouldbeittrue% falses are all in group

% make insert an 'overwrite' and embed default sets
\def\@insertchk#1{\ifx\glsinsert\@empty#1\else\glsinsert\fi}%
\def\thesissidenotebase{% \ifglsshouldbeit\expandafter\textit\fi
{\ifdefempty\glscustomtext{%
\glsifplural{%
  \glscapscase{%
    \@insertchk{\glsentryplural{\glslabel}}%
  }{%
    \@insertchk{\Glsentryplural{\glslabel}}%
  }{%
    \mfirstucMakeUppercase{\@insertchk{\glsentryplural{\glslabel}}}%
  }%
}{%
  \glscapscase{%
    \@insertchk{\glsentrytext{\glslabel}}%
  }{%
    \@insertchk{\Glsentrytext{\glslabel}}%
  }{%
    \mfirstucMakeUppercase{\@insertchk{\glsentrytext{\glslabel}}}%
  }%
}%
}{\@insertchk{\glscustomtext}}}}

\def\thesissidenotesymbol{%
   \ifglsshouldbeit\expandafter\textsl\fi{\@insertchk{\glssymbol{\glslabel}}}%
}
\def\thesis@sidesymbol#1{%
   \disablesidefalse\glslink{#1}{\textbf{\glsentrytext{#1}}}:\disablesidetrue\space\thesis@shortgls
}
\robustify\thesis@sidesymbol

\newif\ifthesisglsinside \thesisglsinsidetrue

\def\GlsResetSide#1{\checkoddpage\csundef{gls@lock@#1@\number\oddpage@page}}

% at beginning of every math mode, define an empty list
% we use these lock-lists as math mode typesets twice with out a reliable way to detect whether we are in teh first or second pass.
% with this we counteract cases in which the same symbol appears multiple time in the math mode ~> we only trigger for the first
% use within each math mode (and have such a list to escape scopes added by the user)
\appto\mathopen{\global\let\@gls@locked@list\@empty}
% do the same for inline math

\newif\if@glss@isin@locklist
% #1 label #2 when #3 then
\def\@gls@isin@locklist#1#2#3{%
   \@glss@isin@locklistfalse
   \protected@edef\@check{#1}%
   \@for\@tmp@lock@list:=\@gls@locked@list\do{%
      \ifx\@tmp@lock@list\@check
         \@glss@isin@locklisttrue
      \fi
   }%
   \if@glss@isin@locklist#2\else#2\fi
}



% #1 will only be executed if not locked; #2 is the label; #3 is else
\def\thesisgls@lockForPage#1#2#3{\ifthesisglsinside% we identify the exact page:
   \checkoddpage% used so i do not have to fiddle with page label
   % we lock the current label for the page; in math mode we have to do so twice to account for pre typeset calculations
\ifmmode
   A \csname @gls@locked@list\endcsname B
   \ifcsname @gls@lock@#2@\number\oddpage@page @@\endcsname
      % if the character is already in the list, we yield #3
      \@gls@isin@locklist{#2}{#3}{%
      \xappto\@gls@locked@list{{#2}}%
      \ifcsname gls@lock@#2@\number\oddpage@page\endcsname
      {#3}%
      \else
      \expandafter\gdef\csname gls@lock@#2@\number\oddpage@page\endcsname{}\relax
      #1\fi}
   \else 
      \expandafter\gdef\csname @gls@lock@#2@\number\oddpage@page @@\endcsname{}\relax
      % append to the list
      \xappto\@gls@locked@list{{#2}}%
      #1
   \fi
\else
   \ifcsname gls@lock@#2@\number\oddpage@page\endcsname
   {#3}%
   \else
   \expandafter\gdef\csname gls@lock@#2@\number\oddpage@page\endcsname{}\relax
   #1\fi
\fi
\else
   #3%
\fi}


\def\thesissidenotegls{\ifinsidenote\else
   \global\glsshouldbeittrue
   % disableside to restrict (may-be infinite) propagation
   \thesisgls@lockForPage{\thesissidenotebase\sidenote{\protect\thesis@sidesymbol{\glslabel}}}{\glslabel}{\global\glsshouldbeitfalse\thesissidenotebase}%
   \fi
}
\robustify\thesissidenotegls
\def\thesissidenoteGls{%
   \global\glsshouldbeittrue\thesisgls@lockForPage{\thesissidenotebase\sidenote{\protect\disablesidefalse\protect\glslink{\glslabel}{\textbf{\capitalisewords{\glsentrytext{\glslabel}}}:\protect\disablesidetrue\protect\space\thesis@shortgls}}}{\glslabel}{\global\glsshouldbeitfalse\thesissidenotebase}%
}
\robustify\thesissidenoteGls
\def\thesissidenoteSymbol{%
   \global\glsshouldbeittrue\thesisgls@lockForPage{\thesissidenotesymbol\sidenote{\protect\disablesidefalse\protect\glslink{\glslabel}{\textbf{{\glssymbol{\glslabel}}}}:\protect\disablesidetrue\protect\space\thesis@shortgls}}{\glslabel}{\global\glsshouldbeitfalse\thesissidenotesymbol}%
}
\robustify\thesissidenoteSymbol

% thesissidenotebase deals with proper plural handling :)
\AtBeginDocument{\defglsentryfmt[\acronymtype]{%
   {\ifdisableside\global\glsshouldbeitfalse\thesissidenotebase\else\thesissidenotegls\fi}%
}
\defglsentryfmt{%
   {\ifdisableside\global\glsshouldbeitfalse\thesissidenotebase\else\thesissidenoteGls\fi}%
}}
\newglossary[ntsl]{notation}{nts}{nto}{Notations}
\defglsentryfmt[notation]{%
   \ifdisableside\global\glsshouldbeitfalse\thesissidenotesymbol\else\thesissidenoteSymbol\fi
}

\def\sidesymbol#1{%
   \begingroup\def\glslabel{#1}\ifdisableside\else
   \global\glsshouldbeittrue\thesisgls@lockForPage{\sidenote{\protect\disablesidefalse\protect\glslink{\glslabel}{\textbf{{\glssymbol{\glslabel}}}}:\protect\disablesidetrue\protect\space\thesis@shortgls}}{\glslabel}{}%
   \fi\endgroup
}

\def\glsac#1{\glsentrylong{#1} (\gls{#1})}
\def\glsplac#1{\glsentrylongpl{#1} (\glspl{#1})}
\def\noside#1{\begingroup\disablesidetrue#1\endgroup}

\renewcommand*\glsdohyperlink[2]{%
   \@glsshowtarget{#1}\thesis@disablehyper{\hyperlink{#1}{#2}}%
}
\glsenablehyper
\def\lglos@spage{\translate{on-page}}\def\lglos@ppage{\translate{on-pages}}

% \paletteA
\RequirePackage{needspace}

\def\thesisstyle#1{%
   \par\needspace{2.25\baselineskip}\strut{%
      % \raisebox{\dimexpr.5\ht\strutbox-.5pt}
      \large\textrm{#1}\thinspace{\raisebox{.5\height}{\tikz{\draw[thick,line cap=round] (0,0) -- (1,0);}}}%
   }%
   \smallskip\par
}
\SetTheGlossaryStyle{\thesisstyle}