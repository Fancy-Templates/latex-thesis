\setchaptertoc
\chapter{Introduction}\label{chp:introduction}
\csummary{What is the problem solved by this thesis?\csumnext What are our research questions?\csumnext What are our contributions?}

\section{Motivation}

\begin{minted}{R}
x <- list(i='hey', j='there')
for(i in x) {
   print(paste("x$i", x$i))
}
print(x)
\end{minted}


To interpret the program above, we take the set of \gls{not:N}, \gls{not:Z}, and \gls{not:R} because they describe funny numbers that we can use to index what we call an \gls{ast}. Now we can use a \gls{cpo} or verify the \gls{def:behavioral-equivalence} when interpreting the code as a \gls{formal:model}.

\section{Problem Statement}

\section{Contributions}

\section{Overview}