\setchaptertoc
\chapter{Introduction}\label{chp:introduction}
\csummary{What is the problem solved by this thesis?\csumnext What are our research questions?\csumnext What are our contributions?}

\section{Motivation}

\begin{minted}{R}
x <- list(i='hey', j='there')
for(i in x) {
   print(paste("x$i", x$i))
}
print(x)
\end{minted}


To interpret the program above, we take the set of \gls{not:N}, \gls{not:Z}, and \gls{not:R} because they describe funny numbers that we can use to index what we call an \gls{ast} (if you want the long form use \texttt{\textbackslash glsac}: \glsac{ast}). Now we can use a \gls{cpo} or verify the \gls{def:behavioral-equivalence} when interpreting the code as a \gls{formal:model}.
\gls{not:N} will not be repeated on this page!

Using the \texttt{\textbackslash glssym} macro defined in the preamble you can even color symbol uses such as \glssym{not:N} automatically if this is to your liking:

\begin{align}
   \glssym{not:N} &= \{1, 2, \ldots\} \\
   \glssym{not:Z} &= \glssym{not:N} \cup \{0\} \cup \{-1, -2, \ldots\} \\
   \glssym{not:R} &= \textit{Set of real numbers} \\
   \gls{not:C} &= \textit{Set of complex numbers} \\
   \glssym{not:N} &\subset \glssym{not:Z} \subset \glssym{not:R} % \subset \glssym{not:C}
\end{align}
% \gls{not:C}

\section{Problem Statement}

\section{Contributions}

\section{Overview}